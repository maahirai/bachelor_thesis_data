\chapter{{2×3ミキサーを用いた液滴移動の\bout{ない}混合手順の生成}}
\label{proposed}
\section{アルゴリズムの概要}
\ref{proposed}~章では,本論文の提案手法である2$\times$3ミキサーを用いた液滴移動の無いPMD上での混合手順の生成手法の処理の流れを擬似コードを交えて説明する.


まず,本論文の提案手法の入出力について説明する.
本論文の提案手法の入力は,図~\ref{fig:inputoutput}(a)の,2$\times$2ミキサーノードと2$\times$3ミキサーノード,試薬液滴ノードの3種類のノードをで構成されている希釈木である.
また,本論文の提案手法の出力は,図~\ref{fig:inputoutput}の(b)から(e)の,入力希釈木に対応したPMD上でのミキサーの混合手順である.
図~\ref{fig:inputoutput}の(b)から(e)において,同じ図内のミキサーは,混合されるタイミング(タイムステップ)が同じミキサー同士である.

図~\ref{fig:inputoutput}の(b)から(e)の混合手順において,図~\ref{fig:inputoutput}(a)の希釈木内で子ノードに対応する試薬液滴やミキサーは,親ミキサーの配置されるセルに提供液滴を残す.
親ミキサーは,子ミキサーや子試薬が残した液滴を移動させずに配置された場所で混合する.

\begin{figure}[tbp]
 \centering\includegraphics[scale=0.41]{img/OutPut.pdf}
 \caption{提案手法の入出力}\label{fig:inputoutput}
\end{figure}

\begin{algorithm}[tbp]
 \caption{提案手法の処理の流れ}\label{alg:all}
 \begin{algorithmic}[1]
     \Require $\mathit{Tree}$:2$\times$2ミキサーノードと2$\times$3ミキサーノード,試薬液滴ノードを含む希釈木 
     \Require $\mathit{PMD}$:$\mathit{Tree}$を配置するのに十分な大きさを持つPMD  
     \State $\mathit{TransformedTree} \gets$ transform($Tree$) \Comment{希釈木の変形}\label{transform_pseudo}
     \State $\mathit{MixInfo \gets}$xntm($\mathit{PMD,TransformedTree}$)  \label{xntm_pseudo}

      \Return $\mathit{MixInfo}$
 \end{algorithmic}
\end{algorithm}
Algorithm~\ref{alg:all}に本論文の提案手法のアルゴリズム全体の擬似コードを示した.
本論文の提案手法において行われる処理は大きく分けると,希釈木の変形操作と,PMD上での液滴移動のない混合手順の生成の2つである.

1つ目の処理である,Algorithm~\ref{alg:all}の\ref{transform_pseudo}行目に記述されている希釈木の変形操作の説明を行う.
希釈木の変形操作は,希釈木を入力とし,変形された希釈木を出力する.
変形操作を高さ3の希釈木に対して行った場合の例を図~\ref{fig:Transform}に示した.
図~\ref{fig:Transform}(a)は変形前の希釈木,図~\ref{fig:Transform}(b)は変形後の希釈木である.

希釈木の変形操作では,まず,希釈木内の各ミキサーノードに,自身をルートとする部分木内に含まれるミキサーノードの個数(予測混雑度)を割り当てる.
その後,親ミキサーノードと,その全ての子ミキサーノードとの間に張られているエッジの順番を,予測混雑度をキー値にして降順でソートする. 

この変形操作の目的を,後の混合手順の生成処理の説明の際にも詳しく述べるが,ここでも簡単に説明する.
混合手順の生成プログラムでは,希釈木のルートのミキサーノードからBFS順で,PMD上での配置先を決めていく.
したがって,予測混雑度をキー値にしたエッジの降順でのソートで,予測混雑度が高い部分木のルートのミキサーノードほど,BFS順において早くPMDへの配置先の決定を行うことができる.
この仕組みにより,予測混雑度の高い部分木のルートのミキサーノードほど,PMD上で空いたセルが多い,つまり,PMD上での配置先の選択肢の多い,タイムステップの小さいときに配置を行える.
また,PMD上への配置が難しい,2$\times$3ミキサーが多くの子を持つ場合でも,予測混雑度に基づく希釈木の変形操作により,タイムステップが小さいときに配置できる. 

以上で述べた通り,希釈木の変形操作では,予測混雑度という評価値に基づいて,どのような変形がなされるかが決められる.

説明のために,それぞれの子ミキサーノードが親ミキサーノードに提供する液滴数を$\mathit{ProvNum}$(子試薬液滴ノードの場合は0),それぞれの子ノードを根とする部分木内のミキサーノード数を$\mathit{MixerNum}$,特殊ケースの加算値を$\mathit{ShouldPlaceFirstly}$,ミキサー$N$の予測混雑度の値を$ECV(N)$とする.
図\ref{fig:TransDetail}\textcircled{\scriptsize 1}の変形は,$ECV(1)=ProvNum+MixerNum=1+3=4$,$ECV(2)=ProvNum+MixerNum=4+3=7$,$ECV(3)=ProvNum+MixerNum=1+3=4$であるため,安定ソートで$M2,M1,M3$という順番に並べ替えられている.
図\ref{fig:TransDetail}\textcircled{\scriptsize 2}の変形は,$ECV(4)=ProvNum+MixerNum=0+1=1$,$ECV(5)=ProvNum+MixerNum+ShouldPlaceFirstly=0+1+100000000=100000001$であるため,安定ソートで$M5,M4$という順番に並べ替えられている.
$ECV(5)$における,$\mathit{ShouldPlaceFirstly}$は,親ミキサーに2$\times$3ミキサーを持つ際に$ProvNum$の値に5を持ったり,親ミキサーに2$\times$2ミキサーを持つ際に$ProvNum$の値に3を持つ場合に加算される.この条件に当てはまる部分木は,配置を優先しなければ,混合手順の生成処理が必ず失敗するため,大きな値を予測混雑度に加えることで,無理やり配置を優先している.
詳しい説明は,混合手順の生成の説明の際に記述する.
図\ref{fig:TransDetail}\textcircled{\scriptsize 3}の変形は,$ECV(6)=ProvNum+MixerNum=2+1=3$,$ECV(7)=ProvNum+MixerNum=2+1=3$であるため,安定ソートで$M6,M7,R3$という順番に並べ替えられている.

\begin{figure}[tbp]
 \centering\includegraphics[scale=0.5]{img/Transform.pdf}
 \caption{希釈木の変形操作}\label{fig:Transform}
\end{figure}


2つ目の処理は,擬似コード内で
\ref{xntm_pseudo}行目に記述された2$\times$3ミキサーを用いた液滴移動のない混合手順の生成である.
図~\ref{fig:inputoutput}(a)の希釈木と初期状態のPMDを入力とし,図~\ref{fig:inputoutput}(b)から(g)の混合手順を出力する.



\begin{algorithm}[tbp]
 \caption{xntmの擬似コード}\label{alg:prop}
 \begin{algorithmic}[1]
     \Require $\mathit{Tree}$:2$\times$2ミキサーノードと2$\times$3ミキサーノード,試薬液滴ノードを含む希釈木 
     \Require $\mathit{PMD}$:$\mathit{Tree}$を配置するのに十分な大きさを持つPMD  
     \State $\mathit{TransformedTree} \gets$ transform($Tree$) \Comment{希釈木の変形}\label{transform_pseudo}
     \State $\mathit{ret} \gets $List() \Comment{混合手順を記録するためのリスト}\label{xntm_pseudo}
     \State $\mathit{OnPMD}$ $\gets$ List() \Comment{PMD上に配置されているモジュールの管理をするリスト}\label{xntm}
     \State $\mathit{Process,TimeForFlushing}\,\gets $  PlaceOnPMD($\mathit{child,PMD}$)
     \State $\mathit{ret}$.append($\mathit{Process}$)
     \State $\mathit{OnPMD}$.append($TransformedTree.RootMixer$)
    
    \State \While{$TransformedTree.RootMixer.state \neq Mixed $ }
        \State$\mathit{mixer} \gets  \mathit{OnPMD}$.pop(0)
        \State $\mathit{AllChildrenPlaced}\gets\mathit{True}$
        \State $\mathit{AllChildrenMixerMixed}\gets\mathit{True}$
        \ForAll{$\mathit{child} \gets \mathit{mixer.Children}$} 
            \If {$\mathit{child}\:\mathit{\mathbf{not}}\,\mathit{\mathbf{in}}\:\mathit{OnPMD}$}
                \State$\mathit{AllChildrenPlaced \gets False}$
                \State $\mathit{TimeForFlushing}\, \gets$  isTimeForFlushing($\mathit{child,PMD}$) 
                \If{$\mathit{TimeForFlushing}$}
                    \State $\mathit{Process}\gets$Flush($\mathit{PMD,OnPMD}$)
                    \State $\mathit{ret}$.append($\mathit{Process}$)
                \EndIf
                \State $\mathit{Process}\, \gets$  PlaceOnPMD($\mathit{child,PMD}$)
                \State $\mathit{ret}$.append($\mathit{Process}$)
                \State $\mathit{OnPMD}$.append($\mathit{child}$) 
            \EndIf 
            \If {$\mathit{child.kind}==\mathit{Mixer} \: \mathit{\mathbf{and}}\: \mathit{child.state}\neq\mathit{Mixed}$}
                \State $\mathit{AllChildrenMixerMixed}\gets\mathit{False}$
            \EndIf 
        \EndFor 
        \If{$\mathit{AllChildrenPlaced}\,\mathit{\mathbf{and}}\, \mathit{AllChildrenMixerMixed}$}
            \State$\mathit{Process}\gets$Mix($\mathit{mixer,PMD}$)
            \State $\mathit{ret}$.append($\mathit{Process}$)
            \State$\mathit{mixer.state} \gets \mathit{Mixed}$ 
        \EndIf
        \State$\mathit{OnPMD}$.append($\mathit{mixer}$)
    \EndWhile 

     \Return $\mathit{ret}$
 \end{algorithmic}
\end{algorithm}

\section{ライブラリを用いたモジュールの配置{アルゴリズム}}
%ライブラリを用いたモジュールの配置方法を例,図や擬似コードを用いながら説明する.
\section{入力希釈木の変形アルゴリズム}


%希釈木の変形操作の手順や目的などを例や擬似コードを用いながら説明する.

