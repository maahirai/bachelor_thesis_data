\chapter{{2×3ミキサーを用いた液滴移動の\bout{ない}混合手順の生成}}
\section{アルゴリズムの概要}
\label{overview}
\ref{overview}~節では,本論文の提案手法である2$\times$3ミキサーを用いた液滴移動の無いPMD上での混合手順の生成アルゴリズムの処理の流れを擬似コードを交えて説明する.

Algorithm~\ref{alg:prop}に本論文の提案手法のプログラム全体の擬似コードを示した.
\begin{algorithm}[tbp]
 \caption{プログラム全体の処理の流れ}\label{alg:prop}
 \begin{algorithmic}[1]
    \Require $\mathit{Tree}$:2$\times$2ミキサーノードと2$\times$3ミキサーノードを含む入力希釈木
     \State $TransformedTree \gets$ transform($Tree$) \Comment{希釈木の変形}
     \Require $\mathit{PMD}$:$\mathit{Tree}$を配置するのに十分な大きさを持つPMD  
     \State $\mathit{OnPMD}$ = List() \Comment{PMD上に配置されているモジュールの管理をするリスト}
     \State $\mathit{OnPMD}$.append($TransformedTree.RootMixer$)
    \State \While{$TransformTree.RootMixer.state \neq Mixed $ }
        $\mathit{mixer} \gets  \mathit{OnPMD}$.pop()
        \State $\mathit{AllChildrenPlaced}=\mathit{True}$
        \State $\mathit{AllChildrenMixerMixed}=\mathit{True}$
        \ForAll{$\mathit{child} \gets \mathit{mixer.Children}$} 
            \If {$\mathit{child}\:\mathit{not}\,\mathit{in}\:\mathit{OnPMD}$}
                $\mathit{AllChildrenPlaced \gets False}$
                \State PlaceOnPMD($\mathit{child,PMD}$)
                \State $\mathit{OnPMD}$.append($\mathit{child}$) 
            \EndIf 
            \If {$\mathit{child.kind}==\mathit{Mixer} \: \mathit{and}\: \mathit{child.state}\neq\mathit{Mixed}$}
                \State $\mathit{AllChildrenMixerMixed}=\mathit{False}$
            \EndIf 
        \EndFor 
        \If{$\mathit{AllChildrenPlaced}\,\mathit{and}\, \mathit{AllChildrenMixerMixed}$}
            Mix($\mathit{mixer,PMD}$)
            $\mathit{mixer.state} = \mathit{Mixed}$ \\
            $\mathit{OnPMD}$.append($\mathit{mixer}$)
        \EndIf
    \EndWhile 
  %\State ifはこんな感じ
  %\State \Return \True
  %\ElsIf{ぼけぼけ}
  %\State \Return \False
  %\EndIf

  %\For{$i=1$ to N}
  %\State forはもちろん
  %\EndFor

  %\ForAll{外部出力リストのゲート\textit{po}}
  %\State forallや
  %\EndFor

  %\Repeat
  %\State リピートも使えます
  %\State $C$内のそれぞれのゲートの出力論理を計算
  %\State OptimizeCircuitUsingCSPF($C$)
  %\Until{$C$の構成に変更なし}

  %\State \Return $C$
 \end{algorithmic}
\end{algorithm}
\section{入力希釈木への変形アルゴリズム}
希釈木の変形操作の手順や目的などを例や擬似コードを用いながら説明する.
\section{ライブラリを用いたモジュールの配置{アルゴリズム}}
ライブラリを用いたモジュールの配置方法を例,図や擬似コードを用いながら説明する.
