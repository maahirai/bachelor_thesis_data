\chapter{{2×3ミキサーを用いた液滴移動の\bout{ない}混合手順の生成}}
\label{proposed}
\section{アルゴリズムの概要}
\ref{proposed}~章では,本論文の提案手法である2$\times$3ミキサーを用いた液滴移動の無いPMD上での混合手順の生成手法の処理の流れを擬似コードを交えて説明する.


まず,本論文の提案手法の入出力について説明する.
本論文の提案手法の入力は,図~\ref{fig:inputoutput}(a)の,2$\times$2ミキサーノードと2$\times$3ミキサーノード,試薬液滴ノードの3種類のノードをで構成されている希釈木である.
また,本論文の提案手法の出力は,図~\ref{fig:inputoutput}の(b)から(e)の,入力希釈木に対応したPMD上でのミキサーの混合手順である.
図~\ref{fig:inputoutput}の(b)から(e)において,同じ図内のミキサーは,混合されるタイミング(タイムステップ)が同じミキサー同士である.

図~\ref{fig:inputoutput}の(b)から(e)の混合手順において,図~\ref{fig:inputoutput}(a)の希釈木内で子ノードに対応する試薬液滴やミキサーは,親ミキサーの配置されるセルに提供液滴を残す.
親ミキサーは,子ミキサーや子試薬が残した液滴を移動させずに配置された場所で混合する.

\begin{figure}[tbp]
 \centering\includegraphics[scale=0.41]{img/OutPut.pdf}
 \caption{提案手法の入出力}\label{fig:inputoutput}
\end{figure}

\begin{algorithm}[tbp]
 \caption{提案手法の処理の流れ}\label{alg:all}
 \begin{algorithmic}[1]
     \Require $\mathit{Tree}$:2$\times$2ミキサーノードと2$\times$3ミキサーノード,試薬液滴ノードを含む希釈木 
     \Require $\mathit{PMDSize}$:使用するPMDのサイズ
     \State $\mathit{TransformedTree} \gets$ transform($Tree$) \Comment{希釈木の変形操作}\label{transform_pseudo}
     \State $\mathit{MixInfo \gets}$genMixOrder($\mathit{TransformedTree,PMDSize}$) \Comment{混合手順の生成} \label{xntm_pseudo}

      \Return $\mathit{MixInfo}$
 \end{algorithmic}
\end{algorithm}
Algorithm~\ref{alg:all}に本論文の提案手法のアルゴリズム全体の擬似コードを示した.
本論文の提案手法は大きく分けると,希釈木の変形操作と,PMD上での液滴移動のない混合手順の生成の,2つの処理によって構成されている.

\section{入力希釈木の変形アルゴリズム}
まず,1つ目の処理である,Algorithm~\ref{alg:all}の\ref{transform_pseudo}行目の,希釈木の変形操作の説明を行う. 希釈木の変形操作は,希釈木を入力とし,変形された希釈木を出力する. 変形操作を高さ3の希釈木に対して行った場合の例を図~\ref{fig:Transform}に示した.
図~\ref{fig:Transform}(a)は変形前の希釈木,図~\ref{fig:Transform}(b)は変形後の希釈木である.

希釈木の変形操作では,まず,希釈木内の各ミキサーノードに,自身をルートとする部分木内に含まれるミキサーノードの個数(予測混雑度)を割り当てる.
その後,親ミキサーノードと,その全ての子ミキサーノードとの間に張られているエッジの順番を,予測混雑度をキー値にして降順でソートする. 

具体例を用いて,希釈木の変形操作で行われる処理を説明する.
ミキサーノード$N$の予測混雑度の値をECV($N$),ミキサーノード$N$をルートとする部分木内のミキサーノード数をMixerNum($\mathit{N}$)とする.
図~\ref{fig:Transform}において,ECV(1) = MixerNum(1) = 2,ECV(2) = MixerNum(2) = 2,ECV(3) = MixerNum(3) = 3,ECV(4) = MixerNum(4) = 3である.
ECV(1),ECV(2)$<$ECV(3),ECV(4)であるため,図~\ref{fig:Transform}での希釈木の変形においては,M1,M2をルートとした部分木とM3,M4をルートとした部分木の位置が入れ替えられている.

%次に,この変形操作の目的を説明する.
あるミキサーの子ミキサーや子試薬液滴をPMD上へ配置する際に,その配置先のセルにはすでに試薬液滴やミキサー,中間液滴が配置されている場合があり,それをオーバーラップと呼ぶ.
オーバーラップが発生した場合,配置しようとしていたミキサーや試薬液滴のPMD上への配置を行うタイミングを変更する操作や,以降のタイムステップでの混合に使われないPMD上の中間液滴を水で洗い流して,他の子ミキサーや子試薬を配置するためのセルを空ける操作(フラッシング)が必要になる.
フラッシングの回数が増えるほど,試薬合成で必要になる試薬量は増える.

予測混雑度の定義からも分かる通り,高い予測混雑度を持つミキサーノードは子孫ノードを多く持つ.
したがって,高い予測混雑度を持つミキサーノードをルートとする部分木内のミキサーノードは,多くのオーバーラップを発生させやすい.
また,2$\times$2ミキサーノードは最大4個子ノードを持つのに対して,2$\times$3ミキサーノードは最大6個の子ノードを持つ.
したがって,2$\times$3ミキサーノードは,2$\times$2ミキサーノードよりも,高い予測混雑度を持ちやすく,多くのオーバーラップを発生させやすい.

混合手順の生成処理では,希釈木のルートのミキサーノードからBFS順で,PMD上での配置先を決めていく.
したがって,予測混雑度をキー値にしたエッジの降順でのソートを行うことで,予測混雑度が高いミキサーノードほど,BFS順において早くPMDへの配置先の決定を行うことができるようになる.
この仕組みにより,予測混雑度の高いミキサーノードほど,PMD上で空いたセルが多く,配置方法の自由度が高い,タイムステップが小さい時点での配置が行える.
これにより,オーバーラップに伴うフラッシングの回数を減らすことができる.

\begin{figure}[tbp]
 \centering\includegraphics[scale=0.5]{img/Transform.pdf}
 \caption{希釈木の変形操作}\label{fig:Transform}
\end{figure}

\begin{figure}[tbp]
 \centering\includegraphics[scale=0.5]{img/result.pdf}
    \caption{図~\ref{fig:Transform}(b)の希釈木に対するPMD上での液滴移動のない混合手順の生成結果}\label{fig:result}
\end{figure}

\begin{figure}[tbp]
 \centering\includegraphics[scale=0.41]{img/OutPut.pdf}
 \caption{提案手法の入出力}\label{fig:inputoutput}
\end{figure}

\begin{algorithm}[tbp]
 \caption{希釈木の変形操作}\label{alg:transform}
 \begin{algorithmic}[1]
     \Require $\mathit{Tree}$:2$\times$2ミキサーノードと2$\times$3ミキサーノード,試薬液滴ノードを含む希釈木 
     \Function getMixerNum()
     \State $\mathit{q}\gets$list()

      \Return $\mathit{TransformedTree}$
 \end{algorithmic}
\end{algorithm}
次に,2つ目の処理である,Algorithm~\ref{alg:all}の\ref{xntm_pseudo}行目の,PMD上での液滴移動のない混合手順の生成において行う処理の説明を行う.
図~\ref{fig:Transform}(b)の希釈木と,使用するPMDのサイズを入力とし,図~\ref{fig:result}(a)から(e)のPMD上での液滴移動のない混合手順を出力する.
図~\ref{fig:result}(a)から(e)の左下に書かれたTという値は混合手順が実行されるタイムステップ,Fという値はそのタイムステップに至るまでに行われたフラッシングの数を表す.

図~\ref{fig:process}は,図~\ref{fig:Transform}(b)の希釈木を入力とした,7$\times$7サイズのPMD上での液滴移動のない混合手順の生成過程を示す.



\begin{figure}[tbp]
 \centering\includegraphics[scale=0.9]{img/process.pdf}
 \caption{液滴移動のない混合手順の生成過程}\label{fig:process}
\end{figure}

%\newpage
%\section{ライブラリを用いたモジュールの配置{アルゴリズム}}
\section{\mout{PMD上でのミキサーの混合手順の生成アルゴリズム}}
\begin{algorithm}[tbp]
 \caption{液滴移動のない混合手順の生成アルゴリズムの擬似コード}\label{alg:prop}
 \begin{algorithmic}[1]
     \Require $\mathit{Tree}$:2$\times$2ミキサーノードと2$\times$3ミキサーノード,試薬液滴ノードを含む希釈木 
     \Require $\mathit{PMDSize}$:$\mathit{Tree}$を配置するのに十分なPMDの大きさ
     \State $\mathit{TransformedTree} \gets$ transform($Tree$) \Comment{希釈木の変形}\label{transform_pseudo} \State $\mathit{ret} \gets $List() \Comment{混合手順を記録するためのリスト}\label{xntm_pseudo}
     \State $\mathit{OnPMD}$ $\gets$ List() \Comment{PMD上に配置されているミキサーの管理をするためのリスト}\label{xntm}
     \State $\mathit{Process,TimeForFlushing}\,\gets $  PlaceOnPMD($\mathit{TransformedTree.RootMixer,PMD}$)
     \State $\mathit{ret}$.append($\mathit{Process}$)
     \State $\mathit{OnPMD}$.append($TransformedTree.RootMixer$)
    
    \State \While{$TransformedTree.RootMixer.state \neq Mixed $ }
        \State$\mathit{mixer} \gets  \mathit{OnPMD}$.pop(0)
        \State $\mathit{AllChildrenPlaced}\gets\mathit{True}$
        \State $\mathit{AllChildrenMixerMixed}\gets\mathit{True}$
        \ForAll{$\mathit{child} \gets \mathit{mixer.Children}$} 
            \If {$\mathit{child}\:\mathit{\mathbf{not}}\,\mathit{\mathbf{in}}\:\mathit{OnPMD}$}
                \State$\mathit{AllChildrenPlaced \gets False}$
                \State $\mathit{TimeForFlushing}\, \gets$  isTimeForFlushing($\mathit{child,PMD}$) 
                \If{$\mathit{TimeForFlushing}$}
                    \State $\mathit{Process}\gets$Flush($\mathit{PMD,OnPMD}$)
                    \State $\mathit{ret}$.append($\mathit{Process}$)
                \EndIf
                \State $\mathit{Process}\, \gets$  PlaceOnPMD($\mathit{child,PMD}$)
                \State $\mathit{ret}$.append($\mathit{Process}$)
                \State $\mathit{OnPMD}$.append($\mathit{child}$) 
            \EndIf 
            \If {$\mathit{child.kind}==\mathit{Mixer} \: \mathit{\mathbf{and}}\: \mathit{child.state}\neq\mathit{Mixed}$}
                \State $\mathit{AllChildrenMixerMixed}\gets\mathit{False}$
            \EndIf 
        \EndFor 
        \If{$\mathit{AllChildrenPlaced}\,\mathit{\mathbf{and}}\, \mathit{AllChildrenMixerMixed}$}
            \State$\mathit{Process}\gets$Mix($\mathit{mixer,PMD}$)
            \State $\mathit{ret}$.append($\mathit{Process}$)
            \State$\mathit{mixer.state} \gets \mathit{Mixed}$ 
        \EndIf
        \State$\mathit{OnPMD}$.append($\mathit{mixer}$)
    \EndWhile 

     \Return $\mathit{ret}$
 \end{algorithmic}
\end{algorithm}


%ライブラリを用いたモジュールの配置方法を例,図や擬似コードを用いながら説明する.


%希釈木の変形操作の手順や目的などを例や擬似コードを用いながら説明する.

