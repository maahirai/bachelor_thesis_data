\chapter{{2×3ミキサーを用いた液滴移動の\bout{ない}混合手順の生成}}
\label{proposed}
\section{アルゴリズムの概要}
\ref{proposed}~章では,本論文の提案手法である2$\times$3ミキサーを用いた液滴移動の無いPMD上での混合手順の生成手法の処理の流れを擬似コードを交えて説明する.


まず,本論文の提案手法の入出力について説明する.
本論文の提案手法の入力は,図~\ref{fig:inputoutput}(a)の様な,2$\times$2ミキサーノードと2$\times$3ミキサーノード,試薬液滴ノードの3種類のノードをで構成されている希釈木である.
また,本論文の提案手法の出力は,図~\ref{fig:inputoutput}の(b)から(e)の,入力希釈木に対応したPMD上でのミキサーの混合手順である.
図~\ref{fig:inputoutput}の(b)から(e)において,同じ図内のミキサーは混合されるタイミング(タイムステップ)が同じミキサー同士である.

図~\ref{fig:inputoutput}の(b)から(e)の混合手順において,図~\ref{fig:inputoutput}(a)の希釈木内で子ノードに対応する試薬液滴やミキサーは,親ミキサーの配置されるセルに提供液滴を残す.
これによって,本手法においても,NTMと同様にPMD上での試薬液滴の移動のない混合手順の生成が実現されている.

\begin{figure}[tbp]
 \centering\includegraphics[scale=0.45]{img/OutPut.pdf}
 \caption{提案手法の入出力}\label{fig:inputoutput}
\end{figure}

次に,本論文の提案手法のアルゴリズムを擬似コードを交えて説明する.
Algorithm~\ref{alg:prop}に本論文の提案手法のアルゴリズム全体の擬似コードを示した.
本論文の提案手法において行われる処理は大きく分けると,擬似コード内で
\ref{xntm_pseudo}行目以降に記述された2$\times$3ミキサーを用いた液滴移動のない混合手順の生成と,
\ref{transform_pseudo}行目に記述されている入力された希釈木への変形操作の2つである.


希釈木の変形操作では,図~\ref{fig:Transform}(a)のような2$\times$2ミキサーノード,2$\times$3ミキサーノード,
試薬液滴ノードの3種類のノードを含む希釈木内の各ミキサーノードが,その子ミキサーノードと子試薬液滴ノード(子ノード)との間に張られているエッジの順番を,その子ノードを根とする部分木の予測混雑度をキー値にして降順にソートする.

\begin{figure}[tbp]
 \centering\includegraphics[scale=0.5]{img/Transform.pdf}
 \caption{希釈木の変形操作}\label{fig:Transform}
\end{figure}
\begin{figure}[tbp]
 \centering\includegraphics[scale=0.5]{img/TransformInDetail.pdf}
 \caption{希釈木の変形された部分}\label{fig:TransDetail}
\end{figure}

\begin{algorithm}[tbp]
 \caption{提案手法の処理の流れ}\label{alg:prop}
 \begin{algorithmic}[1]
     \Require $\mathit{Tree}$:2$\times$2ミキサーノードと2$\times$3ミキサーノード,試薬液滴ノードを含む希釈木 
     \Require $\mathit{PMD}$:$\mathit{Tree}$を配置するのに十分な大きさを持つPMD  
     \State $\mathit{TransformedTree} \gets$ transform($Tree$) \Comment{希釈木の変形}\label{transform_pseudo}
     \State $\mathit{ret} \gets $List() \Comment{混合手順を記録するためのリスト}\label{xntm_pseudo}
     \State $\mathit{OnPMD}$ $\gets$ List() \Comment{PMD上に配置されているモジュールの管理をするリスト}\label{xntm}
     \State $\mathit{Process,TimeForFlushing}\,\gets $  PlaceOnPMD($\mathit{child,PMD}$)
     \State $\mathit{ret}$.append($\mathit{Process}$)
     \State $\mathit{OnPMD}$.append($TransformedTree.RootMixer$)
    
    \State \While{$TransformedTree.RootMixer.state \neq Mixed $ }
        \State$\mathit{mixer} \gets  \mathit{OnPMD}$.pop(0)
        \State $\mathit{AllChildrenPlaced}\gets\mathit{True}$
        \State $\mathit{AllChildrenMixerMixed}\gets\mathit{True}$
        \ForAll{$\mathit{child} \gets \mathit{mixer.Children}$} 
            \If {$\mathit{child}\:\mathit{\mathbf{not}}\,\mathit{\mathbf{in}}\:\mathit{OnPMD}$}
                \State$\mathit{AllChildrenPlaced \gets False}$
                \State $\mathit{TimeForFlushing}\, \gets$  isTimeForFlushing($\mathit{child,PMD}$) 
                \If{$\mathit{TimeForFlushing}$}
                    \State $\mathit{Process}\gets$Flush($\mathit{PMD,OnPMD}$)
                    \State $\mathit{ret}$.append($\mathit{Process}$)
                \EndIf
                \State $\mathit{Process}\, \gets$  PlaceOnPMD($\mathit{child,PMD}$)
                \State $\mathit{ret}$.append($\mathit{Process}$)
                \State $\mathit{OnPMD}$.append($\mathit{child}$) 
            \EndIf 
            \If {$\mathit{child.kind}==\mathit{Mixer} \: \mathit{\mathbf{and}}\: \mathit{child.state}\neq\mathit{Mixed}$}
                \State $\mathit{AllChildrenMixerMixed}\gets\mathit{False}$
            \EndIf 
        \EndFor 
        \If{$\mathit{AllChildrenPlaced}\,\mathit{\mathbf{and}}\, \mathit{AllChildrenMixerMixed}$}
            \State$\mathit{Process}\gets$Mix($\mathit{mixer,PMD}$)
            \State $\mathit{ret}$.append($\mathit{Process}$)
            \State$\mathit{mixer.state} \gets \mathit{Mixed}$ 
        \EndIf
        \State$\mathit{OnPMD}$.append($\mathit{mixer}$)
    \EndWhile 

     \Return $\mathit{ret}$
 \end{algorithmic}
\end{algorithm}

\section{ライブラリを用いたモジュールの配置{アルゴリズム}}
%ライブラリを用いたモジュールの配置方法を例,図や擬似コードを用いながら説明する.
\section{入力希釈木の変形アルゴリズム}


%希釈木の変形操作の手順や目的などを例や擬似コードを用いながら説明する.

