\chapter{おわりに}
%\begin{itemize}
%\item 本論文の概要と特徴
%\item 得られた成果
%\item それから得られる最終結論
%\item 残された課題
%\end{itemize}
%を書く.
バイオチップの一種であるPMDは,バルブで囲まれたセルという領域が格子状に並んだ構造を持つ.
PMDでは,バルブの開閉により,ミキサーの混合を行う.
また,PMDでは,セル間の液滴の移動は原則不可能である.
この対策として,PMD上で試薬合成を行う際は,セル間の液滴移動のない混合手順を取る必要がある.
しかし,既存のセル間の液滴移動のない混合手順生成手法(NTM)の扱うことができる入力データ(希釈木)の種類は,2$\times$2ミキサーノードのみを含む希釈木と,かなり限定されていた.
したがって,本論文では2$\times$2ミキサーのみでなく,2$\times$3ミキサーも用いたPMD上での液滴移動のない混合手順の生成手法を提案した.

また,2$\times$3ミキサーを扱うにあたって,PMD上でミキサー同士の配置セルが重なった状態(オーバーラップ)の発生や,オーバーラップに対する対処(フラッシング)を行う
回数の増加を抑える必要がある.
本論文では,このオーバーラップやフラッシングの増加への対処として,予測混雑度と呼ばれるミキサーノードの評価値を用いて,希釈木へ変形操作を行う手法を提案した.

実験では,複数の高さの希釈木に関して,変形操作を行った場合と行わなかった場合,
それぞれの希釈木をPMD上での液滴移動のない混合手順の生成手法の入力として与え,
その出力の混合手順内で必要になるフラッシングの回数を比較した.
この比較の結果として,希釈木の変形操作を行うことによる,混合手順内で必要になるフラッシングの回数の平均削減率が算出された.

平均削減率の値より,希釈木の高さが大きいほど,変形操作による混合手順内で必要になるフラッシング数の削減効果は高くなることが分かった.
また,高さの小さい希釈木に対する変形操作による,混合手順内で必要になるフラッシング数の削減効果は低いことも分かった.

最後に,残された課題を述べる.
高さの小さい希釈木に対する変形操作における,混合手順内で必要になるフラッシング数の削減効果を高めるためには,
予測混雑度の算出方法を変更する必要があると考えられる.
したがって,
高さの小さい希釈木に対する変形操作においても,高いフラッシング数の削減効果を得られるような予測混雑度の算出法を求めることは,
今後の課題である.
また,本論文内では詳しくは書いていないが,ミキサーのPMD上での配置先セルの決定は現在ヒューリスティックで行っている.
配置先セルの決定を行うヒューリスティックの改善を行えば,さらなるフラッシング数の削減を行える可能性もある.
したがって,このヒューリスティックの改善も,今後の課題である.


