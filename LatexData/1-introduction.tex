\chapter{はじめに}
近年, 生化学分野において,実験室規模で行われていた従来の\bout{実験方法に代わる新たな実験装置}としてバイオチップが注目されている.
そして,バイオチップの一種にPMD(Programmable Microfluidic Device)がある~\cite{1}
PMDは,長方形のチップ上に多くのセルが配置された構造を持つ.
セルとセルの間はバルブで区切られている.
\bout{PMDは,バルブの開閉によってミキサーと呼ばれる環状の流路を生成し,そこでセル内に配置された液滴の混合操作を行う.}
この特性から,PMDは試薬合成で用いられる~\cite{2}.
PMDが使用する試薬液滴量は,ナノ・ピコリッター単位と非常に少ない.
そのため,PMDを用いて試薬合成を行えば,試薬合成で必要となる試薬量を減らすことができるという利点がある~\cite{1}.

PMDには欠点もある.
それは,PMDのセル内に配置された液滴を他のセルに移動させることが原則的に不可能なこと\redout{である}~\cite{FU2015343}.
\redout{この欠点を補うために},PMD上での\bout{セル間の液滴の移動のない}混合手順の生成手法NTM(No Transport Mixing)が提案されている~\cite{4}.
\bout{NTMは,希釈木を入力とし,その希釈木に対応したPMD上でのセル間の液滴の移動のない混合手順を出力する}.

\bout{NTM}が\bout{扱うことの}できる希釈木の種類は限られている.
\cout{その理由は,}\bout{NTM}が\gout{2$\times$2のセルを用いるミキサーを表す}2$\times$2ミキサー\bout{ノード}のみを含む希釈木しか\bout{扱うことができない}\cout{からである.}
それに対して,2$\times$2ミキサー\bout{ノード}に加え,\gout{2$\times$3のセルを用いるミキサーを表す}2$\times$3ミキサー\bout{ノード}も含む希釈木を扱う場合,\bout{扱う}ことができる希釈木の種類は格段に増える.
それにも関わらず,2$\times$3ミキサー\bout{ノード}も含む希釈木を扱う混合手順の生成手法\redout{は}これまで提案され\cout{ていない.}
\mout{それは,2$\times$3ミキサーの配置は2$\times$2ミキサーの配置よりも難しいからである.}

\gout{あるミキサーの混合に必要な中間結果の液滴を生成するミキサーを子ミキサーという.}
\bout{希釈木内で}2$\times$3ミキサー\bout{ノード}は多くの子ミキサー\bout{ノード}を持つ場合がある.
\gout{その場合,すべての子ミキサーを同時に\cout{配置}することができない,オーバーラップ\cout{という状態になる}可能性が高い.オーバーラップが発生している場合,配置できる子ミキサーでの混合を実行したあと,その中間結果の液滴を洗い流して(フラッシング),他の子ミキサーを\cout{配置}するためのセルを確保する必要がある.}
\cout{したがって,2$\times$2ミキサーより多くの子を持ち,フラッシングの回数が多い2$\times$3ミキサーの配置は難しい.}

\cout{2$\times$3ミキサーノードを含む希釈木を扱うためには,混合手順の生成中に発生するフラッシングの回数を減らす必要がある.}
\cout{前述した通り,}オーバーラップの発生回数が増加するほど,フラッシングの回数も増加する.したがって,\redout{フラッシングの回数を減らすためには,オーバーラップの発生回数を減らせば良い}. それを踏まえて,本論文で提案する手法では,入力された希釈木に対して変形操作を行った後に混合手順の生成を行う.
\redout{
希釈木への変形操作は,希釈木内の全ミキサーについて,それぞれの子ノードを予測混雑度という評価値が左から右へ降順となるように並び替える.
予測混雑度とは,あるノードを根とする部分木内のミキサー\bout{ノードに対応するPMD上のミキサー}が将来どれくらい多くのオーバーラップを発生させるか予測する値である.
希釈木への変形操作は,予測混雑度が高いノードほどPMD上での\cout{配置}先の選択肢が多くなるように並び替える.
それにより,予測混雑度の高いミキサー同士が近接せざるを得ない状況になることを防ぐことが容易になる.
}
したがって,変形操作により,オーバーラップの発生回数が減り,少ないフラッシング回数で2$\times$3ミキサー\bout{ノード}を含んでいる希釈木に対しても混合手順\redout{の}生成を行うことができるようになる.

本論文の実験では,変形操作を行った場合と変形操作を行わなかった場合のそれぞれの希釈木を,混合手順\rout{の}生成手法の入力とした.
そして,\rout{それぞれの}混合手順\rout{の}生成処理中に必要となるフラッシングの回数で手法の評価をした.
その結果,\gout{高さ5の希釈木では,}希釈木に変形操作を行わなかった場合と比較して,希釈木に変形操作を行った場合は平均\gout{4.96}\%フラッシングの回数が少なかった.
\redout{実験結果より},希釈木への変形操作が混合手順生成処理中に発生するフラッシングの回数を減らすことに寄与すること\redout{が確認された}.

本論文は,全5章で構成されている.章構成は以下の通りである.
第2章では,PMDの概要と既存手法について述べる.
第3章では,本論文の提案手法について述べる.
第4章では,提案手法の評価方法とその結果について述べる.
第5章では,まとめと今後の課題について述べる.

%\begin{itemize}
% % デフォルトの箇条書きは項目間や段落間のスペースが広いので下記のように調整した方が綺麗に見えるかも
% 
% \setlength{\parskip}{0cm} % 段落間
% \setlength{\itemsep}{0cm} % 項目間
% \item どのような分野の研究か,その背景について説明する.
%%     \begin{itemize}
%%        \item 近年, 生化学分野において実験室規模で行われていた従来の実験に代わる新たな実験の装置としてバイオチップが注目されている.
%%        \item バイオチップの一種にPMD(Programmable Microfluidic Device)がある
%%        \item PMDでは,バルブの開閉によってセル内の液滴を制御し,液滴の移動や混合を実現する. %%    \end{itemize} % % \item その分野の従来の研究状況について説明する. %    %%     \begin{itemize}
%%        \item PMDには,セル間での液滴の移動が原則的に不可能という欠点がある.したがって,その欠点を補い,セル間の移動無しで試薬の混合を行う試薬混合手法NTM(No Transport Mixing)が提案されている.
%%    \end{itemize
% \item そして,何が解決すべき問題(本論文で扱った問題)かを説明.
%%     \begin{itemize}
%%        \item NTMを実現する既存の希釈木のPMDへの混合手順の生成法は,2$\times$2ミキサーのみを含む希釈木にしか対応していない.したがって,入力として与えることができる希釈木の種類は限られている.
%%        {\item 2$\times$3ミキサーを用いると,多くの子ミキサーを持つ場合がある.その場合,子ミキサーどうしのオーバーラップが発生しやすいため,ミキサーの配置が難しい.}
%%    \end{itemize}
% \item どのようなアイデアで解決したか,キーアイデアを少しだけ披露
%    % \begin{itemize}
%    %     \item {希釈木が2$\times$3ミキサーを含んでいた場合も液滴の移動無しでの混合手順を生成できるよう混合手順生成手法に工夫を施した.
%    %        \begin{itemize}
%    %                \item 入力希釈木に対して変形操作を行う.
%    %                \item PMD上へのミキサーの配置順を変更する.
%    %        \end{itemize}}
%    %\end{itemize}
% \item どのような(実験)結果が得られたか、アピール(目次案の段階では希望的予測)
%%     \begin{itemize}
%%        \item 変形操作を行った後に入力希釈木をPMDへとマッピングすることにより,希釈木の変形を行わなかった場合と比較して,平均17\%のFlushing回数の減少が見られた.
%%     \end{itemize}
%
%\item 章構成
%   % \begin{itemize}
%   % \item 第2章:PMDの概要,{2$\times$2ミキサーを用いた液滴の移動の無い混合手順の生成}
%   % \item 第3章:{2$\times$3ミキサーを用いた液滴の移動の無い混合手順の生成}
%   % \item 第4章:提案手法の評価方法とその結果
%   % \item 第5章:まとめと今後の課題
%   % \end{itemize}
%
%\end{itemize}
