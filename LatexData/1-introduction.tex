\chapter{はじめに}
\begin{itemize}
 % デフォルトの箇条書きは項目間や段落間のスペースが広いので下記のように調整した方が綺麗に見えるかも
 
 \setlength{\parskip}{0cm} % 段落間
 \setlength{\itemsep}{0cm} % 項目間
 \item どのような分野の研究か,その背景について説明する.
     \begin{itemize}
        \item 近年, 生化学分野において実験室規模で行われていた従来の実験に代わる新たな実験の装置としてバイオチップが注目されている.
        \item バイオチップの一種にPMD(Programmable Microfluidic Device)がある.
        \item PMDでは,バルブの開閉によってセル内の液滴を制御し,液滴の移動や混合を実現する.
    \end{itemize}
 \item その分野の従来の研究状況について説明する.
     \begin{itemize}
        \item PMDには,セル間での液滴の移動が原則的に不可能という欠点がある.したがって,その欠点を補い,セル間の移動無しで試薬の混合を行う試薬混合手法NTM(No Transport Mixing)が提案されている.
    \end{itemize}
 \item そして,何が解決すべき問題(本論文で扱った問題)かを説明.
     \begin{itemize}
        \item NTMを実現する既存の希釈木のPMDへの混合手順の決定法は,2×2ミキサーのみを含む希釈木にしか対応していない.したがって,入力として与えることができる希釈木の種類は限られている.
        {\item 2×3ミキサーを用いると,多くの子ミキサーを持つ場合がある.その場合,子ミキサーどうしのオーバーラップが発生しやすいため,ミキサーの配置が難しい.}
    \end{itemize}
 \item どのようなアイデアで解決したか,キーアイデアを少しだけ披露
     \begin{itemize}
         \item {希釈木が2×3ミキサーを含んでいた場合も液滴の移動無しでの混合手順を生成できるよう混合手順生成アルゴリズムに工夫を施した.
            \begin{itemize}
                    \item 入力希釈木に対して変形操作を行う.
                    \item PMD上へのミキサーの配置順を変更する.
            \end{itemize}}
    \end{itemize}
 \item どのような(実験)結果が得られたか、アピール(目次案の段階では希望的予測)
     \begin{itemize}
        \item 変形操作を行った後に入力希釈木をPMDへとマッピングすることにより,希釈木の変形を行わなかった場合と比較して,平均17\%のFlushing回数の減少が見られた.
     \end{itemize}

\item 章構成
    \begin{itemize}
    \item 第2章:PMDの概要,{2×2ミキサーを用いた液滴の移動の無い混合手順の生成}
    \item 第3章:{2×3ミキサーを用いた液滴の移動の無い混合手順の生成}
    \item 第4章:提案手法の評価方法とその結果
    \item 第5章:まとめと今後の課題
    \end{itemize}

\end{itemize}
