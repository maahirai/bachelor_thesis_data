\chapter*{内容梗概}
%\newcommand{\rout}[1]{\textcolor{red}{#1}}
%\newcommand{\bout}[1]{\textcolor{blue}{#1}}
%\newcommand{\gout}[1]{\textcolor{green}{#1}}
%\newcommand{\cout}[1]{\textcolor{cyan}{#1}}
%\newcommand{\mout}[1]{\textcolor{magenta}{#1}}
%\newcommand{\yout}[1]{\textcolor{yellow}{#1}} など色変可
ver2:ver1でalainさんに指摘された点について修正し,その部分を赤色にしました.
あと,第1章でのNTMという単語の説明・用法を,参考文献に忠実になるよう修正しました.

ver3:ver2で山下先生に指摘された点について修正し,その部分を青色にしました.
あと,「入力として受け取る」などまわりくどい表現をしてしまっていたのは,恐らくNTMの入出力関係を明記していなかったためなので,
NTMの名前が出た文の直後にその入出力関係についての文を青色で書きました.
フラッシングの説明も青色で書きました.

ver4:ver3で山下先生に指摘された点について修正し,その部分を緑色にしました.

ver5:ver4で山下先生に指摘された点を,水色にしました.(新たに書いた第二章も水色になってしまっています.すみません.)

ver6:第2章でalainさんに指摘された点について修正したところを,水色にしました.あと,図の説明をする文章に関して「示す」と「表す」が混じっていたので,不自然にならないところは「示す」に水色で直しました.

ver7:ver6で山下先生に指摘された点について修正したところを,ピンク色にしました.あと,第3章を書き,alainさんに指摘された点について修正したところもピンク色にしました.
