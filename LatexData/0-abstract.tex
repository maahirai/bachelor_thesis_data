\chapter*{内容梗概}
%\newcommand{\rout}[1]{\textcolor{red}{#1}}
%\newcommand{\bout}[1]{\textcolor{blue}{#1}}
%\newcommand{\gout}[1]{\textcolor{green}{#1}}
%\newcommand{\cout}[1]{\textcolor{cyan}{#1}}
%\newcommand{\mout}[1]{\textcolor{magenta}{#1}}
%\newcommand{\yout}[1]{\textcolor{yellow}{#1}} など色変可
ver2:ver1でalainさんに指摘された点について修正し,その部分を赤色にしました.
あと,第1章でのNTMという単語の説明・用法を,参考文献に忠実になるよう修正しました.

ver3:ver2で山下先生に指摘された点について修正し,その部分を青色にしました.
あと,「入力として受け取る」などまわりくどい表現をしてしまっていたのは,恐らくNTMの入出力関係を明記していなかったためなので,
NTMの名前が出た文の直後にその入出力関係についての文を青色で書きました.
フラッシングの説明も青色で書きました.

ver4:ver3で山下先生に指摘された点について修正し,その部分を緑色にしました.

ver5:ver4で山下先生に指摘された点を,水色にしました.(新たに書いた第二章も水色になってしまっています.すみません.)

ver6:第2章でalainさんに指摘された点について修正したところを,水色にしました.あと,図の説明をする文章に関して「示す」と「表す」が混じっていたので,不自然にならないところは「示す」に水色で直しました.

ver7:ver6で山下先生に指摘された点について修正したところを,ピンク色にしました.
第3章を書きました.
あと,2章最後の方の「子ノード」についてなのですが,3章でも使っているので少し書き換えて残しました.

ver8:
ver7で,alainさんに指摘された点について修正したところをピンク色にしました.目次案のときから少し節で書く内容を変えました.そこもピンクにしてます.

ver9:
ver8で,山下先生に指摘された点について修正したところを青色にしました.

ver10: 
ver9で,山下先生に指摘された点について修正したところを赤色にしました.
第4章を書きました.

ver11: 
ver10で,alainさんに指摘された点を赤色で修正しました.

ver12: 
ver11で,山下先生に指摘された点を緑色で修正しました.アブストと第5章を書きました.
(チェックを受け次第変更履歴はコメントアウトします)

ver13: 
ver12で,alainさんに指摘された点を緑色で修正しました.図や表も修正しました.
(このバージョン以降では,変更履歴はコメントアウトします)
\newpage 

近年研究されているバイオチップの\gout{内の1種類に},PMD(Programmable Microfluidic Device)がある.
PMDは,バルブという部品に囲まれたセルという領域が格子状に並んだ構造を持つバイオチップである.
PMDでは,バルブの開閉によって作られる,ミキサーと呼ばれる環状の流路を用いることで,液滴の混合を行うことができる.
また,PMDでは,ミキサーでの試薬液滴や中間液滴の混合を繰り返すことで,ナノ・ピコリッター程度のごく少量の試薬の使用による,目標濃度の混合液滴の生成(試薬合成)ができる.
生化学実験において,試薬合成で必要になる試薬量は実験全体で必要となる試薬量の約90\%を占め,実験全体で必要となる試薬量の削減においてPMDが果たす役割は大きい.

しかし,PMDで試薬合成を行う際に気をつけるべき点もある.
それは,セル間の液滴の移動はできないという点である.
このPMDの特徴に対処するために,4つのセルを用いるミキサー(2$\times$2ミキサー)を用いたPMD上での液滴移動のない混合手順の生成手法No Transport Mixing(NTM)が提案され\gout{ている}.
しかし,NTMには扱うことができる入力データ(希釈木)の種類が少ないという欠点がある.
したがって,本論文ではNTMの欠点を補うために,2$\times$2ミキサーに加え,6つのセルを用いるミキサー(2$\times$3ミキサー)も用いたPMD上での液滴移動のない混合手順の生成手法を提案する.
しかし,2$\times$3ミキサーには使用における難点がある.
それは,PMD上でのミキサー同士の配置先セルが被った状態(オーバーラップ)を発生させやすいということである.
オーバーラップが発生した場合,不要な混合液\gout{滴}を水で洗い流す操作(フラッシング)を行う必要がある.
フラッシングは,試薬合成で使用する試薬液滴量\gout{を}増加させる.したがって,試薬合成\gout{で行われるフラッシングの}回数は削減する必要がある.

本論文では,試薬合成で行われるフラッシングの回数を削減するために,フラッシングを引き起こすオーバーラップの発生回数を削減する.
本論文の提案手法では,オーバーラップの発生回数を削減するために,希釈木内のミキサーを表すノード(ミキサーノード)の位置を,自身のオーバーラップの
発生させやすさを評価した値(予測混雑度)の大きさに基づいて並び替えることにより,希釈木の変形操作を行った.
希釈木の変形操作では,希釈木の高さが大きくなるほど,高いフラッシングの回数の削減効果が得られた.
最も変形操作によるフラッシングの回数の平均削減率が高かった高さ5の希釈木では,4.96\%の平均削減率が得られた.
